% ---------------------------- Problem 1----------------------------------
\subsubsection*{\center Задача № 1.}
{\bf Условие.~}
Разложить в ряд Фурье заданную функцию $f(x)$, построить графики $f(x)$ и суммы ее ряда Фурье. Если не указывается, какой вид разложения в ряд необходимо представить, то требуетчя разложить функцию либо в общий тригонометрический ряд Фурье, либо следует выбрать оптимальный вид разложения в зависимости от данной функции.

\[
f(x)=\begin{cases}
-1, 	& 0 \leqslant x \leqslant 2,\\
x-3, 	& 2 \leqslant x \leqslant 3.
\end{cases}
\]
{\bf Решение.~}	
%График
\begin{center}
	\begin{tikzpicture}
	\begin{axis}[xmin=-1,	xmax=3.5, 	ymin=-1,	ymax=0.5,
	width=0.5\textwidth,
	height=0.4\textwidth,
	axis x line=middle,
	axis y line=middle, 
	every axis x label/.style={at={(current axis.right of origin)},anchor=west},
	every inner x axis line/.append style={|-latex'},
	every inner y axis line/.append style={|-latex'},
	minor tick num=1,			
	axis equal=true,
	xlabel=$x$, 
	ylabel=$y$,          
	samples=100,
	clip=true,
	]
	\addplot[color=black, line width=1.5pt,domain=0:2] {-1};
	\addplot[color=black, line width=1.5pt,domain=2:3]{\x-3};
	\addplot[thick,dashed] coordinates {(2,0) (2,-1)};
	\addplot[
	mark=*,
	mark options={fill=black, draw=black},
	only marks,
	] coordinates {(2, -1)};
	\end{axis}
	\end{tikzpicture}
\end{center}
\noindent
Построим общий тригонометрический ряд Фурье вида
$$
f(x)=\frac{a_0}{2}+\sum_{n=1}^\infty 
	\left(a_n\cos{(n\omega x)}+b_n\sin{(n\omega x)}\right),\quad\text{где}\,\omega=\frac{2\pi}{T},\,T=3.
$$
\noindent
Вычислим коэффициенты
$$
\begin{array}{rcl}
a_0 &=& \displaystyle\frac{2}{3}\left.\left(
\int\limits_0^2
(-1)\,dx + \int\limits_2^3
(x-3)\,dx \right) = 
\frac{2}{3}\left(
-2 + \left( \frac{x^2}{2}-3x\right)
\right|_2^3 \right) = -\frac{5}{3},												\\[12pt]
a_n &=& \displaystyle\frac{2}{3}\left(
	-\int\limits_0^2
	\cos \frac{2\pi nx}{3}\,dx + \int\limits_2^3
	(x-3)\cos \frac{2\pi nx}{3}\,dx \right) ={}									\\[12pt]
	&=& \displaystyle\frac{2}{3}\left(
	-\left.\frac{3}{2\pi n}\sin\frac{2\pi nx}{3} \right|_0^2
	+\left.\frac{3}{2\pi n}(x-3)\sin\frac{2\pi nx}{3} \right|_2^3 
	+\left.\frac{9}{(2\pi n)^2}(x-3)\cos\frac{2\pi nx}{3} \right|_2^3\right) = 	\\[12pt]
	&=& \displaystyle-\frac{3}{2\pi^2 n^2}\left(1-\cos\frac{4\pi n}{3} \right),	\\[12pt]
b_n &=& \displaystyle\frac{2}{3}\left(
	-\int\limits_0^2
	\sin \frac{2\pi nx}{3}\,dx + \int\limits_2^3
	(x-3)\sin \frac{2\pi nx}{3}\,dx \right) ={}									\\[12pt]
	&=& \displaystyle\frac{2}{3}\left(
	\left.\frac{3}{2\pi n}\cos\frac{2\pi nx}{3} \right|_0^2
	-\left.\frac{3}{2\pi n}(x-3)\cos\frac{2\pi nx}{3} \right|_2^3 
	+\left.\frac{9}{(2\pi n)^2}(x-3)\sin\frac{2\pi nx}{3} \right|_2^3\right) =	\\[12pt]
	&=& \displaystyle-\frac{1}{\pi n}-\frac{3}{2\pi^2 n^2}\sin\frac{4\pi n}{3}.
\end{array}
$$
Применив теорему Дирихле о поточечной сходимости ряда Фурье, видим, что построенный ряд Фурье сходится 
к периодическому (с периодом $T=3$) продолжению исходной функции при всех $x\ne 3n$, и 
$S(3n)=-1/2$ при $n=0,\pm1,\pm2,\ldots$, где $S(x)$ --- сумма ряда Ферье. 
График функции $S(x)$ имеет следующий вид
\begin{center}
	\begin{tikzpicture}
	\begin{axis}[xmin=-6, xmax=6, ymin=-1, ymax=0.5,
	width=0.8\textwidth,
	height=0.4\textwidth,
	axis x line=middle,
	axis y line=middle, 
	every axis x label/.style={at={(current axis.right of origin)},anchor=west},
	every inner x axis line/.append style={|-latex'},
	every inner y axis line/.append style={|-latex'},
	minor tick num=1,			
	axis equal=true,
	xlabel=$x$, 
	ylabel=$S(x)$,          
	samples=100,
	clip=true,
	]
	\addplot[color=black, line width=1.5pt,domain=-6:-4] {-1};
	\addplot[color=black, line width=1.5pt,domain=-4:-3]{\x+3};
	\addplot[color=black, line width=1.5pt,domain=-3:-1] {-1};
	\addplot[color=black, line width=1.5pt,domain=-1:0]{\x};
	\addplot[color=black, line width=1.5pt,domain=0:2] {-1};
	\addplot[color=black, line width=1.5pt,domain=2:3]{\x-3};
	\addplot[color=black, line width=1.5pt,domain=3:5] {-1};
	\addplot[color=black, line width=1.5pt,domain=5:6]{\x-6};
	\addplot[thick,dashed] coordinates {(-4,0) (-4,-1)};
	\addplot[thick,dashed] coordinates {(-1,0) (-1,-1)};
	\addplot[thick,dashed] coordinates {(2,0) (2,-1)};
	\addplot[thick,dashed] coordinates {(5,0) (5,-1)};
	\addplot[
	mark=*,
	mark options={fill=black, draw=black},
	only marks,
	] coordinates {(-6, -0.5) (-3, -0.5) (0, -0.5) (3, -0.5) (6, -0.5)};
	\end{axis}
	\end{tikzpicture}
\end{center}
\noindent
\textbf{Ответ:}
\[
\begin{split}
&f(x)=-\frac{5}{6}+\sum_{n=1}^\infty\left[ \frac{3}{2\pi^2 n^2}\left(1-\cos\frac{4\pi n}{3} \right)\cos\frac{2\pi nx}{3}-\left(\frac{1}{\pi n}+\frac{3}{2\pi^2 n^2}\sin\frac{4\pi n}{3} \right)\sin\frac{2\pi nx}{3}\right], x\ne 3n; \\
&S(3n)=-\frac{1}{2}, \text{ при } n\in\mathbb{Z}.
\end{split}
\]




% ---------------------------- Problem 2----------------------------------
\subsubsection*{\center Задача № 2.}
{\bf Условие.~}
Для заданной графически функции $y(x)$ построить ряд Фурье в комплексной форме, изобразить график суммы построенного ряда

%График
\begin{center}
	\begin{tikzpicture}
	\begin{axis}[xmin=-1,	xmax=3.5, 	ymin=-1,	ymax=0.5,
	width=0.5\textwidth,
	height=0.4\textwidth,
	axis x line=middle,
	axis y line=middle, 
	every axis x label/.style={at={(current axis.right of origin)},anchor=west},
	every inner x axis line/.append style={|-latex'},
	every inner y axis line/.append style={|-latex'},
	minor tick num=1,			
	axis equal=true,
	xlabel=$x$, 
	ylabel=$y$,          
	samples=100,
	clip=true,
	]
	\addplot[color=black, line width=1.5pt,domain=0:2] {-1};
	\addplot[color=black, line width=1.5pt,domain=2:3]{\x-3};
	\addplot[thick,dashed] coordinates {(2,0) (2,-1)};
	\addplot[
	mark=*,
	mark options={fill=black, draw=black},
	only marks,
	] coordinates {(2, -1)};
	\end{axis}
	\end{tikzpicture}
\end{center}

\noindent
\textbf{Решение.}\\

\noindent
Ряд Фурье в комплексной форме имеет следующий вид
\[
f(x) = \sum_{n=-\infty}^\infty c_n e^{i\omega nx},\quad c_n=\frac{1}{T}\int\limits_a^b f(x) e^{-i\omega nx}dx,\,\omega=\frac{2\pi}{T}.
\]
В нашем примере $ a=0,b=3,T=3,\omega=2\pi/3$, 
найдем коэффицинеты $c_n,\,n=0,\pm1,\pm2,\ldots$
где $\omega=2\pi/T,\,T=3.$
$$
\begin{array}{rcl}
c_0 &=&\displaystyle\frac{1}{3} \int\limits_0^3 f(x)dx=\frac{a_0}{2}=-\frac{5}{6},\\[12pt]
c_n &=&\displaystyle\frac{1}{3}\left(
-\int\limits_0^2
e^{-i\omega nx}dx+ \int\limits_2^3
(x-3) e^{-i\omega nx}dx \right) ={}\\[12pt]
&=&\displaystyle\frac{1}{3}\left(
-\left.\frac{3i}{2\pi n} e^{-i\omega nx} \right|_0^2
+\left.\frac{3i}{2\pi n}\left[(x-3) e^{-i\omega nx} - \frac{3i}{2\pi n} e^{-i\omega nx} \right]\right|_2^3\right) = \\[12pt]
&=&\displaystyle\frac{i}{2\pi n}+\frac{3}{4\pi^2 n^2}\left(1- e^{-2\omega ni} \right)=\frac{3}{4\pi^2 n^2}\left(1-\cos\frac{4\pi n}{3} \right)+\frac{i}{2\pi n}\left(1+\frac{3}{2\pi n}\sin\frac{4\pi n}{3} \right).
\end{array}
$$
\noindent
Применив теорему Дирихле о поточечной сходимости ряда Фурье, видим, что построенный ряд Фурье сходится 
к периодическому (с периодом $T=3$) продолжению исходной функции при всех $x\ne 3n$, и $S(3n)=-1/2$ при 
$n=0,\pm1,\pm2,\ldots$, где $S(x)$ --- сумма ряда Фурье. График функции $S(x)$ имеет вид
\begin{center}
	\begin{tikzpicture}
	\begin{axis}[xmin=-6, xmax=6, ymin=-1, ymax=0.5,
	width=0.8\textwidth,
	height=0.4\textwidth,
	axis x line=middle,
	axis y line=middle, 
	every axis x label/.style={at={(current axis.right of origin)},anchor=west},
	every inner x axis line/.append style={|-latex'},
	every inner y axis line/.append style={|-latex'},
	minor tick num=1,			
	axis equal=true,
	xlabel=$x$, 
	ylabel=$S(x)$,          
	samples=100,
	clip=true,
	]
	\addplot[color=black, line width=1.5pt,domain=-6:-4] {-1};
	\addplot[color=black, line width=1.5pt,domain=-4:-3]{\x+3};
	\addplot[color=black, line width=1.5pt,domain=-3:-1] {-1};
	\addplot[color=black, line width=1.5pt,domain=-1:0]{\x};
	\addplot[color=black, line width=1.5pt,domain=0:2] {-1};
	\addplot[color=black, line width=1.5pt,domain=2:3]{\x-3};
	\addplot[color=black, line width=1.5pt,domain=3:5] {-1};
	\addplot[color=black, line width=1.5pt,domain=5:6]{\x-6};
	\addplot[thick,dashed] coordinates {(-4,0) (-4,-1)};
	\addplot[thick,dashed] coordinates {(-1,0) (-1,-1)};
	\addplot[thick,dashed] coordinates {(2,0) (2,-1)};
	\addplot[thick,dashed] coordinates {(5,0) (5,-1)};
	\addplot[
	mark=*,
	mark options={fill=black, draw=black},
	only marks,
	] coordinates {(-6, -0.5) (-3, -0.5) (0, -0.5) (3, -0.5) (6, -0.5)};
	\end{axis}
	\end{tikzpicture}
\end{center}

\noindent
\textbf{Ответ:}
\[
\begin{split}
&f(x)=\sum_{n=-\infty}^\infty\left[ \frac{3}{4\pi^2 n^2}\left(1-\cos{\frac{4\pi n}{3}} \right)+\frac{i}{2\pi n}\left(1+\frac{3}{2\pi n}\sin{\frac{4\pi n}{3}} \right)\right] e^{\tfrac{i2\pi nx}{3}},~ x\ne 3n; \\
&S(3n)=-\frac{1}{2},\quad\text{при}~n\in\mathbb{Z}.
\end{split}
\]


% ---------------------------- Problem 3----------------------------------
\subsubsection*{\center Задача № 3.}
{\bf Условие.~}\\
Найти резольвенту для интегрального уравнения Вольтерры со следующим ядром 
$$K(x,t)=x.$$

\noindent
{\bf Решение.~}\\
\noindent
Запишем интегральное уравнение Вольтерры
$$
y(x)=x^3+\int\limits_0^x x y(t)dt.
$$
Из рекурентных соотношений получаем
$$
\begin{array}{rcl}
K_1(x,t)&=&\displaystyle x, \\[12pt]
K_2(x,t)&=&\displaystyle\int\limits_t^x K(x,s)K_1(s,t)ds = \int\limits_t^x x s ds = x\cdot\frac{x^2-t^2}{2},\\[12pt]
K_3(x,t)&=&\displaystyle\int\limits_t^x K(x,s)K_2(s,t)ds = \int\limits_t^x x s \frac{s^2-t^2}{2} ds = \frac{x}{2}\left(\frac{x^2-t^2}{2}\right)^2.\\[12pt]
K_j(x,t)&=&\displaystyle\frac{x}{(j-1)!}\left(\frac{x^2-t^2}{2}\right)^{j-1}\!\!\!\!\!\!\!\!,\quad j=\mathbb{N}.
\end{array}
$$
Подставляя это выражение для итерированных ядер, найдем резольвенту
$$ 
R(x,t,\lambda)=x\sum_{j=1}^\infty \frac{\lambda^{j-1}}{(j-1)!}\left(\frac{x^2-t^2}{2}\right)^{j-1}\!\!\!\!\!\!\!\!,
\quad j=1,2,\ldots
$$

